\documentclass[a4paper, 12pt]{article}
\usepackage[utf8]{inputenc}
\usepackage[T1,T2A]{fontenc}
\usepackage[a4paper, top=2cm, bottom=2cm, left=1cm, right=1cm, marginparwidth=1.75cm]{geometry}
\usepackage{graphicx}
\usepackage{amsmath}
\usepackage{indentfirst}
\usepackage[english, russian]{babel}
\usepackage[section,above,below]{placeins}
\usepackage[noend]{algorithmic}
\usepackage{amssymb}
\usepackage{amsfonts}
\usepackage{pdfpages} 
%\usepackage[14pt]{extsizes}


\begin{document}

\section{Определение задачи оптимального управления}

Пусть имеется некоторая динамическая система, состояние которой в
каждый момент времени $t$ описывается вектор-функцией ${\bf x}(t) \in \mathbb{R}^n$. На
состояние системы можно воздействовать, изменяя управляемые параметры
${\bf u}(t) \in  {\bf U}_t \subseteq  \mathbb{R}^r$. Будем рассматривать класс кусочно-непрерывных управлений ${\bf u}(t)$.

При заданном управлении ${\bf u}(t)$ состояние системы изменяется во времени
согласно закону:
\begin{equation}
    \dot{\bf x}={\bf f}(t,{\bf x},{\bf u})
\end{equation}

Рассмотрим задачу оптимального управления данной системой:
определить управление ${\bf u}^*(t)$, доставляющее экстремум критерию качества
вида:
\begin{equation}
    J({\bf x},{\bf u})=\int_{t_0}^{t_1} F(t, {\bf x}(t), {\bf u}(t)) dt + F_0(t_0,t_1,{\bf x}(t_0),{\bf x}(t_1)) \rightarrow \max
\end{equation}

При этом первое слагаемое (интегральная часть критерия) характеризует
качество функционирования системы на всем промежутке управления $[t_0, t_1]$,
тогда как второе слагаемое (терминальный член) –- только конечный
результат воздействия управления, определяемый начальным ${\bf x}(t_0)$ и
конечным ${\bf x}(t_1)$ состояниями и, возможно, моментами начала и окончания
управления $t_0$ и $t_1$. В зависимости от физического смысла задачи
интегральная или терминальная часть критерия может быть равна нулю.

На процесс функционирования системы могут накладываться
дополнительные ограничения в форме краевых условий:
\begin{equation}
    F_i(t_0,t_1,{\bf x}(t_0),{\bf x}(t_1))=0,i=1..m,
\end{equation}
задающие множества допустимых начальных и конечных состояний системы
и моментов начала и окончания управления.
Важным частным случаем (2.3) являются условия вида:
\begin{equation}
    {\bf x}(t_k)={\bf x}_k, k=1,0 ,
\end{equation}
соответствующие закрепленному левому или правому концу фазовой
траектории.

Моменты времени начала и окончания управления, $t_0$ и $t_1$, могут
полагаться как известными, тогда говорят о задаче с фиксированным
временем управления, или неизвестными (задача с нефиксированным
моментом начала или окончания управления).

Необходимые условия оптимальности в данной задаче, точнее,
необходимые условия сильного локального максимума даются принципом
максимума Понтрягина.

{\bf Теорема}. Пусть (${\bf x}^*(t)$, ${\bf u}^*(t)$, $t_0^*$, $t_1^*$) –- оптимальный процесс в предыдущей задаче. Тогда найдутся одновременно не равные нулю множители ${\bf \lambda}=(\lambda_0,\dots,\lambda_m),\lambda_0 \ge 0$ и  
${\bf \psi} (t)=(\psi_1(t),\dots,\psi_n(t))$ такие, что
выполнены следующие условия:
\begin{enumerate}
    \item Функция Понтрягина задачи $H(t,{\bf x}, {\bf u}, {\bf \psi},\lambda_0)=\lambda_0 F(t,{\bf x}, {\bf u}) + ({\bf \psi},{\bf f}(t,{\bf x}, {\bf u}))$ при каждом $t \in [t_0,t_1]$ достигает максимума по ${\bf u}$ в точке ${\bf u}^*(t)$, когда ${\bf x}={\bf x}^*(t)$.
    \item Вектор-функция ${\bf \psi}(t)$ удовлетворяет сопряженной системе дифференциальных уравнений: $${\dot {\bf \psi}}_i (t)=- \dfrac{\partial H(t,{\bf x}^*(t),{\bf u}^*(t),{\bf \psi}(t),\lambda_0)}{\partial x_i}, i=1..n,$$ с краевыми условиями (условия трансверсальности):
    $$\psi_i (t_0^*)=-(\lambda,\dfrac{\partial F_0 (t_0^*,t_1^*,{\bf x}(t_0),{\bf x}(t_1))}{\partial x_i(t_0)} );$$
    $$\psi_i (t_1^*)=-(\lambda,\dfrac{\partial F_0 (t_0^*,t_1^*,{\bf x}(t_0),{\bf x}(t_1))}{\partial x_i(t_1)} ).$$
    \item Выполнены условия на подвижные концы:
    $$H(t,{\bf x}, {\bf u}, {\bf \psi},\lambda_0)|_{t=t_0}=(\lambda,\dfrac{\partial F_0 (t_0^*,t_1^*,{\bf x}(t_0),{\bf x}(t_1))}{\partial t_0} );$$
    $$H(t,{\bf x}, {\bf u}, {\bf \psi},\lambda_0)|_{t=t_1}=-(\lambda,\dfrac{\partial F_0 (t_0^*,t_1^*,{\bf x}(t_0),{\bf x}(t_1))}{\partial t_1} ).$$
\end{enumerate}

{\bf Замечания}.
\begin{enumerate}
    \item Множитель Лагранжа  $\lambda_0$ определяет чувствительность оптимального
решения задачи к виду интегральной части функционала. В вырожденном
случае совокупность ограничений задачи такова, что оптимальное
управление ${\bf u}^*(t)$ не зависит от вида интеграла $F(t, {\bf x}(t), {\bf u}(t))$. При этом из условий принципа максимума следует, что  
$\lambda_0 = 0$. В невырожденном случае $\lambda_0 > 0$, поэтому ее можно положить равной 1 (разделив функцию Н на 0). При этом условия принципа максимума не изменятся.
  \item Для задачи с закрепленными концами сопряженная функция  $\psi (t)$
имеет свободные концы, т.е. соответствующие условия трансверсальности
отсутствуют.
Обратно, для задачи со свободными концами сопряженная функция имеет закрепленные концы, определяемые соотношениями:
$$\psi_i (t_0)=-\dfrac{\partial F_0 (t_0^*,t_1^*,{\bf x}(t_0),{\bf x}(t_1))}{\partial x_i (t_0)} ;$$
$$\psi_i (t_1)=\dfrac{\partial F_0 (t_0^*,t_1^*,{\bf x}(t_0),{\bf x}(t_1))}{\partial x_i (t_1)} .$$
\end{enumerate}

\end{document}