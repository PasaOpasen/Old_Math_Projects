\documentclass[a4paper, 12pt]{article}
\usepackage[utf8]{inputenc}
\usepackage[T1,T2A]{fontenc}
\usepackage[a4paper, top=2cm, bottom=2cm, left=1cm, right=1cm, marginparwidth=1.75cm]{geometry}
\usepackage{graphicx}
\usepackage{amsmath}
\usepackage{indentfirst}
\usepackage[english, russian]{babel}
\usepackage[section,above,below]{placeins}
\usepackage[noend]{algorithmic}
\usepackage{amssymb}
\usepackage{amsfonts}
\usepackage{pdfpages}


\begin{document}
\section{Методы кластерного анализа}
\section{Иерархические алгоритмы}
\section{Расстояния между кластерами}
\section{Процедуры эталонного типа}
\section{Методика дискриминантного анализа}
\section{Что характеризует Лямбда Уилкса?}
\section{Что показывают квадраты расстояний Махаланобиса?}
\section{Какое максимальное число канонических дискриминантных функций допустимо в дискриминантном анализе?}
\section{Какую информацию дают стандартизованные и структурные коэффициенты дискриминантной функции?}
\section{Опишите процедуру отбора переменных с помощью стандартизованных и структурных коэффициентов}
\section{Какова интерпретация канонического коэффициента корреляции?}
\section{В каком случае учет априорных вероятностей может сильно изменить результаты классификации?}
\section{Методика факторного анализа}
\section{Суть задачи вращения общих факторов}
\section{Критерий Кайзера}
\section{Критерий Каменистой осыпи}
\section{Метод главных компонент}

\end{document}